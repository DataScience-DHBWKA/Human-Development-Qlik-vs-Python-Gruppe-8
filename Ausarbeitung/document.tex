\documentclass[12pt]{article}

\usepackage[ngerman]{babel}

\usepackage{geometry}
\geometry{
	a4paper,
	total={170mm,257mm},
	left=35mm,
	right=25mm,
	top=25mm,
	bottom=20mm
}


\renewcommand{\thefootnote}{\Roman{footnote}}

\begin{document}
	\title{Ein Vergleich der Visualisierungsmöglichkeiten in Looker und Python anhand eines Pokemon Datensatzes}
	\author{Daniel Linck, Christof Warsinsky, Kai Schillinger, David Helwich}
	\date{\today}
	
	\maketitle
	
	\newpage
	
	\tableofcontents
	
	\newpage
	
	\section{Einleitung}
	
	\subsubsection{Ziele und Features von Looker Studio laut Hersteller?}
	
	\subsubsection{Überblick Nutzung von Looker Studio}
	blblabla
	
	\section{Vergleich Looker und Python}
	\subsection{Einleitung}
	
	\subsection{Effektivität}
	
	\subsection{Benutzerfreundlichkeit}
	
	\subsection{Flexibilität}

	\subsection{Interaktivität und Skalierbarkeit}
 

	\subsection{Zusammenarbeit und Teilen}

	\subsection{Integration}
	
	\subsection{Barrierefreiheit}

	\subsection{Kosten und Lizenzierung}

	\section{SWOT-Analyse}
	\subsection{Python}

	
	\subsection{Looker}

	\subsection{Welches Tool eignet sich ggf. unter verschiedenen Anwendungsszenarien?}

	
	\section{Fazit}
	
\end{document}