\documentclass[a4paper,12pt]{article}
\usepackage[utf8]{inputenc}
\usepackage[T1]{fontenc}
\usepackage[ngerman]{babel}
\usepackage{graphicx}
\usepackage{setspace}
\usepackage{geometry}
\usepackage{titlesec}
\usepackage{tocloft}

% Seitenränder einstellen
\geometry{left=3cm,right=3cm,top=3cm,bottom=3cm}

% Zeilenabstand einstellen
\onehalfspacing

% Überschriften formatieren
\titleformat{\section}{\normalfont\Large\bfseries}{\thesection}{1em}{}
\titleformat{\subsection}{\normalfont\large\bfseries}{\thesubsection}{1em}{}

% Inhaltsverzeichnis anpassen
\renewcommand{\cftsecleader}{\cftdotfill{\cftdotsep}} % Punkte im Inhaltsverzeichnis
\renewcommand{\contentsname}{Inhaltsverzeichnis} % Umbenennung des Inhaltsverzeichnisses

\begin{document}
	
	% Titelseite
	\begin{titlepage}
		\centering
		{\scshape\LARGE Titel der Arbeit \par}
		\vspace{1cm}
		{\Large\itshape Vorname Nachname\par}
		\vspace{0.5cm}
		{\Large\itshape Vorname2 Nachname2\par}
		\vspace{0.5cm}
		{\Large\itshape Vorname3 Nachname3\par}
		\vspace{2cm}
		{\Large\itshape Betreut von\par}
		Prof. Dr. Betreuer \textsc{Name}
		
		\vfill
		
		% Unterer Teil der Seite
		{\large Datum\par}
	\end{titlepage}
	
	% Inhaltsverzeichnis
	\tableofcontents
	\newpage
	
	% Einleitung
	\section{Einleitung}
	Die Einleitung...
	
	% Hauptteil
	\section{Hauptteil}
	\subsection{Abschnitt 1}
	Text für Abschnitt 1...
	
	\subsection{Abschnitt 2}
	Text für Abschnitt 2...
	
	% Schluss
	\section{Schluss}
	Zusammenfassung und Schlussfolgerungen...
	
	% Literaturverzeichnis
	\begin{thebibliography}{9}
		\bibitem{beispiel1} Autor1, A. (Jahr). Titel der Quelle 1. Verlag.
		\bibitem{beispiel2} Autor2, B. (Jahr). Titel der Quelle 2. Verlag.
	\end{thebibliography}
	
\end{document}
